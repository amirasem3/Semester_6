\documentclass{article}
\usepackage[margin=0.7in]{geometry}
\usepackage{graphicx}
\usepackage{color}
\usepackage{hyperref}
\graphicspath{ {/asay/Desktop/images/} }
\usepackage{xepersian}




\title{پاسخ تمرین شماره ۵ درس معماری کامپیوتر }


\author{امیر حسین عاصم یوسفی \\ ۹۶۱۱۰۳۲۳}
\settextfont{B Nazanin}

\begin{document}
	\maketitle
	
\section*{سوال ۱}
با توجه به این طولانی ترین تاخیر مربوط به 
\lr{Mem}
می باشد که برابر با ۲ نانوثانیه می باشد و این که تاخیر رجیستر های میانی ۰/۱ نانوثانبه هستند پس برای 
\lr{Clock Time}
جدید داریم  : 
\begin{center}
	\lr{Clock Time new = 2ns + 0.1ns = 2.1ns}
\end{center}
برای این که تاثیر
\lr{stall}
را در نظر بگیریم باید 
\lr{CPI}
را برای زمانی که 
\lr{stall}
داریم طبق فرمول زیر به دست آوریم : 
\begin{center}
	\lr{CPI = IdealCPI + stall cycles}
\end{center}
در این جا 
\lr{IdealCPI = 1}
در نظر می گیریم و چون به ازای هر ۴ دستور یک 
\lr{stall}
داریم پس 
\begin{center}
	\lr{stall cycles  = 1/4 = 0.25ns}
\end{center}
بنابراین طبق فرمول گفته شده 
\textcolor{red}{\lr{new CPI (with Stall)  = 1 + 0.25 = 1.25ns}}\\
تا به حال مقادیر 
\lr{CT}
و 
\lr{CPI}
را برای معماری 
\lr{Pipeline}
به دست آوردیم حال  
\lr{Speedup}
راطبق فرمول زیر به دست می آوریم  : 
\begin{center}
 \lr{SpeedUp = $ \frac{Single  \  cycle \  EXE  \ time}{Pipeline \  EXE  \ time }$}\\
 \lr{Execution TIme  = CPI . IC  . Cycle time}
\end{center}
حال با توجه به بالا داریم  : 
\begin{center}
	\lr{SpeedUp = $\frac{IC \times 1 \times 7}{IC \times 1.25 \times 2.1} = \frac{7}{2.625}$ = 2.67 ns}
\end{center}
\hrule
\section*{سوال ۲}
برای این سوال با توجه به قانون 
\lr{Amdahl}
داریم  : 
\begin{center}
	\lr{SpeedUp = $\frac{1}{\frac{P}{N} + S}$}
\end{center}
که در این جا 
\lr{P}
نشان دهنده بخشی است که به صورت 
\lr{Parallel}
نوشته شده و 
\lr{N}
نشان دهنده تعداد پردازنده ها و 
\lr{S}
بیانگر بخشی است که به صورت موازی نوشته نشده است . پس داریم  : 
\begin{center}
	\lr{SpeedUp = $\frac{1}{\frac{x}{4}+(1-x)} = 2 \rightarrow x = \frac{2}{3}$ $\Rightarrow$ x = 66\%}
\end{center}
بنابراین باید ۶۶ درصد از برنامه را به صورت موازی نوشت تا اجرای برنامه بر روی این ۴ پردازنده ۲ برابر سریع تر شود  . 
\hrule
\section*{سوال ۳ }


\end{document}