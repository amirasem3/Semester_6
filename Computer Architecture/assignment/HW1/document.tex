\documentclass{article}
\usepackage{xepersian}
\title{پاسخ تمرین شماره ۱ درس معماری کامپیوتر}
\author{امیر حسین عاصم یوسفی \\ ۹۶۱۱۰۳۲۳}
\settextfont{B Nazanin}


\begin{document}
  \maketitle
  \section*{ سوال ۱  : }
  با توجه به فرمول محاسبه 
  $CPU  \ TIME$
  که برابر است با 
  $ \frac{CPU \  CLOCK \  CYCLE}{ CLOCK \  RATE} = \frac{(Instruction count *CPI)}{Clock  \  Rate}$
  
  برای میتوان تعداد دستورات از هر دو نوع گفته شده را به دست آورد که به شرح زیرمی باشد  : 
  \begin{center}
  	$A  \ in  \ processor  \  p1 : \frac{3}{200} = \frac{IC(A) * CPI}{200} \Rightarrow IC(A) = \frac{3}{CPI} $\\
  	$A  \ in  \ processor  \  p2 : \frac{5}{300} = \frac{IC(A) * CPI}{300} \Rightarrow IC(A) = \frac{5}{CPI} $\\
  	$ B  \ in  \ processor  \  p1 : \frac{4}{200} = \frac{IC(B) * CPI}{200} \Rightarrow IC(B) = \frac{4}{CPI}$\\
  	$ B  \ in  \ processor  \  p2 : \frac{3}{300} = \frac{IC(B) * CPI}{300} \Rightarrow IC(B) = \frac{3}{CPI}$
  \end{center}
حال با توجه به بالا  می توانیم مجموع دستورات از هرنوع را به دست آوریم  : 
 \begin{center}
 	$Sum \ A \  (p1 \  + \  p2)  = A  \ in  \ processor  \  p1 + A  \ in  \ processor  \  p2 = \frac{8}{CPI}$
 	
 \end{center}
\begin{center}
	$ Sum \ B \  (p1 \  + \  p2)  = B  \ in  \ processor  \  p1 + B  \ in  \ processor  \  p2 = \frac{7}{CPI}$
\end{center}
و چون طبق صورت سوال زمان اجراي یک برنامه خاص در هر دو پردازنده یکسان می باشد بنابراین هر دو مقدار 
$CPI$
با یک دیگر برابرند ، پس داریم : 
\begin{center}
	$  \frac{IC(A)}{IC(B)} = \frac{8}{7}$
\end{center}
\hrule

\end{document}