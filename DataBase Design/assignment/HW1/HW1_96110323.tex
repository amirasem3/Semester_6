\documentclass{article}
\usepackage{graphicx}
\graphicspath{ {/asay/Desktop/images/} }
\usepackage{color}
\usepackage{xepersian}


\title{پاسخ تمرین شماره ۱ درس طراحی پایگاه داده }
\author{امیر حسین عاصم یوسفی \\ ۹۶۱۱۰۳۲۳}
\settextfont{B Nazanin}

\begin{document}
	\maketitle
	\section*{سوال ۱  : }
	برای این سوال فرض های زیر در نظر گرفته شده است  ‌: 
	\begin{center}
		\begin{enumerate}
			\item هر اتاق می تواند توسط چند پرستار مراقبت شود 
			\item وجودیت پرونده به وجودیت بیمار بستگی ندارد (پرونده در بایگانی مرکز درمانی باقی می ماند )
			\item هر اتاق شامل چند تخت می باشد  
			\item ظرفیت اتاق های 
			\lr{VIP}
			و معمولی با یک دیگر فرق دارد  . 
		\end{enumerate}
	\end{center}
و برای طراحی موجودیت ها از تکنیک های زیر استفاده شده است   : 
\begin{center}
	\begin{enumerate}
		\item اتاق  : تکنیک تعمیم
		\item کارمند  : تخصیص کامل 
		\item پزشک  : تخصیص کامل 
	\end{enumerate}
\end{center}
با توجه به نمودار 
\lr{EER}
به روابط زیر می رسیم  : 
\begin{center}
	\begin{enumerate}
		\item 	\textcolor{red}{\lr{MN}}  : هر مدیر  
		\lr{N}
		پرستار را کنترل کند  . هر پرستار توسط یک مدیر کنترل می شود . 
		\item \textcolor{red}{\lr{MD}} : هر مدیر 
		\lr{N}
		پزشک را کنترل می کند . هر پزشک توسط یک مدیر کنترل می شود  . 
		\item \textcolor{red}{\lr{PD}} : هر پزشک 
		\lr{N}
		بیمار را ویزیت می کند .  هر بیمار توسط یک پزشک ویزیت می شود  . 
		\item \textcolor{red}{\lr{PMD}} : برای هر بیمار 
		\lr{M}
		دارو تجویز می شود  . هر دارو برای 
		\lr{N}
		بیمار تجویز می شود . هر پزشک 
		\lr{N}
		دارو تجویز می کند . هر دارو توسط 
		\lr{M}
		پزشک تجویز می شود . 
		هر پزشک 
		\lr{N}
		دارو را برای 
		\lr{M}
		بیمار تجویز می کند . 
		\item \textcolor{red}{\lr{PR}}  : به هر بیمار یک اتاق تعلق می گیرد  . هر اتاق به 
		\lr{N}
		بیمار داده می شود  . 
		\item \textcolor{red}{\lr{DR}} : هر پرونده توسط یک کارمند پذیرش ثبت می شود . هر کارمند پذیرش 
		\lr{N}
		پرونده را ثبت می کند . 
		\item \textcolor{red}{\lr{DA}} : هر کارمند بایگانی 
		\lr{N}
		پرونده را بایگانی می کند . هر پرونده توسط یک کارمند بایگانی ثبت می شود  . 
		\item \textcolor{red}{\lr{NR}} : هر پرستار از 
		\lr{N}
		اتاق مراقبت می کند . هر اتاق توسط 
		\lr{M}
		پرستار مراقبت می شود . 
	\end{enumerate}

\end{center}
\section*{سوال ۲ }
برای این سوال فرض های زیر در نظر گرفته شده  : 
\begin{center}
	\begin{enumerate}
		\item برای صورت حساب ها یک 
		\lr{ID}
		در نظر گرفته شده است  . 
		\item برای پرسشنامه ها نیز یک 
		\lr{ID}
		در نظر گرفته شده است  . 
		
		\item پرسشنامه از بخش های زیر تشکیل شده است  : 
		\begin{enumerate}
			\item نظرات
			\item پیشنهادات
			\item انتقادات 
			\item جواب سوال ها 
			\item 
			\lr{ID}
		\end{enumerate}
	\item پرسشنامه یکتاست یعنی تمام معلمین به یک پرسشنامه واحد جواب می دهد  . 
	\end{enumerate}
\end{center}
با توجه به نمودار 
\lr{ER}
به روابط زیر می رسیم  : 
\begin{center}
	\begin{enumerate}
		\item \textcolor{red}{\lr{TR}}: هر معلم می تواند 
		\lr{N}
		اتاق را رزرو کند . هر اتاق به یک معلم اجاره داده می شود . 
		\item \textcolor{red}{\lr{TB}} : هر معلم 
		\lr{N}
		صورت حساب پرداخت می کند . هر صورت حساب توسط یک معلم پرداخت می شود 
		\item \textcolor{red}{\lr{TQ}} : هر معلم یک پرسشنامه را پر می کند . هر پرسشنامه توسط 
		\lr{N}
		معلم پر می شود  . 
		\item \textcolor{red}{\lr{RR}} : هر درخواست توسط یک کارمند پذیرش بررسی می شود  . هر کارمند پذیرش 
		\lr{N}
		درخواست را بررسی می کند . 
		\item \textcolor{red}{\lr{BR}} : هر کارمند پذیرش 
		\lr{N}
		صورت حساب را صادر می کند . هر صورت حساب توسط یک کارمند پذیرش صادر می شود  . 
		\item \textcolor{red}{\lr{BA}} : هر حساب دار 
		\lr{N}
		صورت حساب را تایید می کند  . هر صورت حساب توسط یک حساب دار تایید می شود  . 
		\item \textcolor{red}{\lr{QMB }} : هر مدیر 
		\lr{N}
		صورت حساب را نظارت می کند  . هر صورت حساب توسط یک مدیر نظارت می کند  . هر پرسشنامه توسط یک مدیر نظارت می شود  . هر مدیر 
		\lr{N}
		پرسشنامه را نظارت می کند. 
	\end{enumerate}
\end{center}
\end{document}