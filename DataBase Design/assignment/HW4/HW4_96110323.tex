\documentclass{article}
\usepackage[margin=0.7in]{geometry}
\usepackage{amsmath}
\usepackage{graphicx}
\usepackage{color}
\usepackage{xcolor}
\graphicspath{ {/asay/Desktop/images/} }
\usepackage{xepersian}




\title{پاسخ تمرین شماره ۴ درس طراحی پایگاه داده }

\author{
	امیر حسین عاصم یوسفی \\ 
	۹۶۱۱۰۳۲۳\\
	علیرضا وفایی \\ 
	۹۵۱۰۵۲۹۵\\
	}
\settextfont{B Nazanin}

\begin{document}
	\maketitle
	\section*{سوال ۱}
	برای این سوال ابتدا درصد اتاق های 
	\lr{VIP}
را به دست می آوریم و بعد بر روی آن یک 
\lr{Assertion}
می زنیم  . که به صورت زیر می باشد : 

\begin{center}
	\lr{create assertion check (30 >=(select count(room\_number) / (select count(room\_number) from room where type =1 or type =2 )*100 from room where type = 2);}
\end{center}
\hrule
\section*{سوال ۲ }
برای این سوال باید طوری 
\lr{Trigger }
را بنویسیم که بعد از هر بروزرسانی اجرا شود که به صورت زیر می باشد : 
\begin{flushleft}
	\lr{create trigger teacher-room  \\
		after update of state\\
		on reserve\\
		referencing old as ostate, new as nstate\\
		for each row \\
		(when state = 'finished' \\
		insert to POLL value (reserve.teacher\_id , reserve.room\_number)}
\end{flushleft}
\hrule
\section*{سوال ۳}
\textcolor{red}{شماره ۱ }
\\
با توجه به جداول داده شده اگر بر روی دو جدول عملگر 
\lr{Join}
را اجرا کنیم 
\lr{P1}
به دلیل این که هیچ محصولی تولید نکرده حذف می شود  . 
\\
\textcolor{red}{شماره ۲  }
\\
برای سازنده 
\lr{P4}
هیچ اسمی در نظر گرفته نشده است 
\\
\textcolor{red}{شماره ۳}
\\
برای محصولات 
\lr{P2}
هیج اسمی در نظر گرفته نشده است  . 
\hrule

\section*{سوال ۴ }
دو رابطه 
\lr{R(A,B)}
و 
\lr{S(B)}
را در نظر بگیرید . حاصل عملیات
\lr{$R \div S$}
برابر است با تمام مقادیر ستون 
\lr{A}
در 
\lr{R}
که با تمام مقادیر ستون 
\lr{B}
در 
\lr{S}
، در ارتباط هستند  . ادعا می کنیم  : 
\begin{center}
	\lr{$ \prod \ _{<A>}(R) \ - \ \prod \ _{<A>}(\prod \ _{<A>}(R) \ \times   \ S  \ - \ R)$}
\end{center}
دلیل : عبارت 
\lr{$ \prod \ _{<A>}(R) \times S $}
شامل تمام جفت های ممکن از مقادیر 
\lr{A}
و 
\lr{B}
به صورت 
\lr{(A , B)}
می باشد   . \\
عبارت 
\lr{$ \prod \ _{<A>}(R) \times S \ - \ R$}
شامل همه جفت های 
\lr{(A, B)}
است که در رابطه 
\lr{R}
نیستند . بنابراین اعمال عملگر پرتو 
\lr{$\prod \ _{<A>}$}
روی این عبارت ، مقادیر 
\lr{A}
را می دهد که با تمام مقادیر 
\lr{B}
در ارتباط نیستند  . 
\\
بنابراین 
\lr{$ \prod \ _{<A>}(R) \ - \ \prod \ _{<A>}(\prod \ _{<A>}(R) \ \times   \ S  \ - \ R)$}
مقادیر 
\lr{A}
را که با تمام مقادیر 
\lr{B}
در ارتباط هستند به دست می دهد  . 
\hrule
\section*{سوال ۵}
\subsection*{سیستم مدیریت مرکز درمانی}
\subsubsection*{\textcolor{red}{الف}}
برای این مورد از یک عملگر 
\lr{join}
استفاده می کنیم . که دستور آن به صورت زیر می باشد : 
\begin{center}
\lr{$\prod \ _{<fname>} (nurse \bowtie _{(supervision.nurse
		\_id = nurse.id \  \wedge \ supervision.room \_number!=2) }supervision)$}
\end{center}
\subsubsection*{\textcolor{red}{ب}}
	\begin{center}
		\lr{$\prod \ _{<doctor\_id \ , \ room\_number>} (visit) = R_1$}\\
		\lr{$\prod \ _{room\_number} (room)  = R_2 $}\\
		\lr{$R_3 = R_1 \div R_2 $}\\
		\lr{$\prod \ _{<fname>} (R_3 \bowtie _{(R_3.doctor\_id = doctor.id)} doctor) $}
	\end{center}

\subsubsection*{\textcolor{red}{پ}}
\begin{center}
	\lr{$ \prod \ _{< nurse\_id \ , \ room\_number>} (supervision) = R_1$}\\
		\lr{$\prod \ _{room\_number} (room)  = R_2 $}\\
	\lr{$R_3 = R_1 \div R_2 $}\\
		\lr{$\prod \ _{<fname>} (R_3 \bowtie _{(R_3.nurse\_id = nurse.id)} nurse) $}
\end{center}
\subsubsection*{\textcolor{red}{ت}}
\begin{center}
	\lr{$\prod \ _{< nurse\_id \ , \ \textcolor{blue}{COUNT}(room\_number) \ as  \ \textcolor{red}{'rn'}>} (supervision) = R_1$}\\
	\lr{$\prod \ _{<fname>} (R_1 \bowtie _{(R_1.nurse\_id = nurse.id \  \wedge \ R_1 .rn \ \geq 2 )} nurse)$}
\end{center}
\subsection*{سیستم رزرو اتاق}
\subsubsection*{\textcolor{cyan}{الف}}
\begin{center}
	\lr{$\prod \ _{<teacher\_id \ , \textcolor{blue}{datediff}(\textcolor{blue}{Getdate()} \ , \ birthday) \ as \textcolor{red}{'Age'}>}(teacher) = R_1$}\\
	\lr{$\prod \ _{<\textcolor{blue}{MAX}(Age) \ as \ \textcolor{red}{'Age'}>} (R_1) = R_2$}
		\lr{$R_3 = R_1 \div R_2 $}\\
		\lr{$\prod \ _{
			<firest\_name>} (R_3 \ \bowtie \ _{(R_3.teacher\_id = teacher.teacher\_id)} \  teacher )$}
\end{center}
\subsubsection*{\textcolor{cyan}{ب}}
\begin{center}
	\lr{$\sigma _{<last\_name \ = \ \textcolor{red}{'Karami'}>} \ (tearcher) = R_1$}\\
	\lr{$\prod \ _{<first\_name>}(R_1 \ \bowtie  \ _{(R_1.salary  \  = \ teacher.salary)} teacher)$}
\end{center}
\subsubsection*{\textcolor{cyan}{پ}}
\begin{center}
	\lr{$ \prod \ _{<room\_number>} (reserve) = R_1$}\\
	\lr{$ \prod \  _{<room\_number>}(room) = R_2$}\\
	\lr{$ \prod \ _{<room\_number>}(R_2  \ SEMIMINUS \ R_1)$}
\end{center}
\subsubsection*{\textcolor{cyan}{ت}}
\begin{center}
	\lr{$ \sigma _{<state \ = \ \textcolor{red}{'Finished'} \  \vee \ state \ = \ \textcolor{red}{'Approved'} \vee \ state \ = \ \textcolor{red}{'Rejected'}\vee \ state \ = \ \textcolor{red}{'New'}\vee \ state \ = \ \textcolor{red}{'Pending'}>} \ (reserve) = R_1$}\\
	\lr{$ \prod \ _{<room\_number \ , \ \textcolor{blue}{COUNT}(room\_number) \ as \ \textcolor{red}{'rn'}>}R_1 = R_2$}\\
	\lr{$\prod \ _{<room\_number \ , \ \textcolor{blue}{MAX}(rn) \ as \ \textcolor{red}{'rn'}>}R_2 = R_3$}\\
	\lr{$ \prod \ _{<room\_number>}(R_3 \ \bowtie\ _{(R_3.rn  \ = \ R_2.rn) } R_2)$}
\end{center}
\hrule
\section*{سوال ۶}
برای مثال رابطه 
\lr{R}
را به صورت زیر تعریف می کنیم  : 
\begin{center}
	\lr{$ \prod \ _{<state \  , \ teacher\_id>} (reserve) = R$}
\end{center}
و رابطه 
\lr{S}
را به صورت زیر تعریف کنیم  : 
\begin{center}
	\lr{$ \prod \ _{<teacher\_id \ , \ first\_name>} (teacher) = S$}
\end{center}
بنابراین قسمت الف به صورت زیر می باشد : 
\begin{center}
	\lr{$\prod \ _{<state \ , \ teacher\_id>} (R \bowtie S) = A$}
\end{center}
که رابطه 
\lr{A}
شامل دو ستون به نام های 
\lr{teacher\_id  , state}
می باشد . 
\\
اگر همین تبدیلات را برای قسمت ب انجام دهیم به رابطه 
\lr{B}
می رسیم که شامل ستون های 
\lr{state , teacher\_id , state , teacher\_id}
می باشد  . 
\\
بنابراین می توان نتیجه گرفت که پاسخ قسمت الف یک زیر مجموعه افقی از پاسخ قسمت ب می باشد  . 
\end{document}