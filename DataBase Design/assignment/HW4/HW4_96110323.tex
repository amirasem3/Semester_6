\documentclass{article}
\usepackage[margin=0.5in]{geometry}
\usepackage{graphicx}
\usepackage{color}
\usepackage{hyperref}
\graphicspath{ {/asay/Desktop/images/} }
\usepackage{xepersian}




\title{پاسخ تمرین شماره ۵ درس طراحی پایگاه داده }


\author{امیر حسین عاصم یوسفی \\ ۹۶۱۱۰۳۲۳}
\settextfont{B Nazanin}

\begin{document}
	\maketitle
	\section*{سوال ۱}
	برای این سوال ابتدا درصد اتاق های 
	\lr{VIP}
را به دست می آوریم و بعد بر روی آن یک 
\lr{Assertion}
می زنیم  . که به صورت زیر می باشد : 

\begin{center}
	\lr{create assertion check (30 >=(select count(room\_number) / (select count(room\_number) from room where type =1 or type =2 )*100 from room where type = 2);}
\end{center}
\hrule
\section*{سوال ۲ }
برای این سوال باید طوری 
\lr{Trigger }
را بنویسیم که بعد از هر بروزرسانی اجرا شود که به صورت زیر می باشد : 
\begin{flushleft}
	\lr{create trigger teacher-room  \\
		after update of state\\
		on reserve\\
		referencing old as ostate, new as nstate\\
		for each row \\
		(when state = 'finished' \\
		insert to POLL value (reserve.teacher\_id , reserve.room\_number)}
\end{flushleft}
\hrule
\section*{سوال ۳}
\textcolor{red}{شماره ۱ }
\\
با توجه به جداول داده شده اگر بر روی دو جدول عملگر 
\lr{Join}
را اجرا کنیم 
\lr{P1}
به دلیل این که هیچ محصولی تولید نکرده حذف می شود  . 
\\
\textcolor{red}{شماره ۲  }
\\
برای سازنده 
\lr{P4}
هیچ اسمی در نظر گرفته نشده است 
\\
\textcolor{red}{شماره ۳}
\\
برای محصولات 
\lr{P2}
هیج اسمی در نظر گرفته نشده است  . 
\hrule

\section*{سوال ۴ }
	
\end{document}