\documentclass{article}
\usepackage{graphicx}
\graphicspath{ {/asay/Desktop/images/} }
\usepackage{color}
\usepackage{xcolor}
\usepackage{xepersian}


\title{پاسخ تمرین شماره ۲ درس طراحی پایگاه داده }
\author{امیر حسین عاصم یوسفی \\ ۹۶۱۱۰۳۲۳}
\settextfont{B Nazanin}

\begin{document}
	\maketitle
	
	\section*{تمرین اول : }
	\subsection*{مرکز درمانی : }
	برای موجودیت ها جدول ها به شکل زیر می باشد  :‌
	\begin{center}
		\begin{enumerate}
			\item بیمار  :  پرونده ، سن ، نام ، نام خانوادگی ، جنسیت ، تاریخ تولد ، 
			\underline{کد ملی }
			،شماره بیمه 
			\item  اقوام بیمار :‌
			\underline{شماره اقوام }
			،نام خانوادگی ، تاریخ  ملاقات ، 
			کلید خارجی (
	\underline{کد ملی بیمار })
	
			\item ویزیت بیمار : تاریخ مراجعه ، 
			\underline{شماره ویزیت }
			، نام بیماری ، تاریخ بستری ، 
			(کلید های خارجی  : \underline{شماره اتاق}  ،\underline{‌کد طبابت}  ، \underline{کد ملی بیمار} ) 

			\item اتاق بستری  : 
			\underline{شماره اتاق }
			، 
		نوع اتاق 
		\item پرستار : شیفت ، 
		\underline{کد پرستاری }
		، جنسیت ، نام خانوادگی ، نام 
		\item تذکر :
		\underline{\lr{ID} }
		، تاریخ ، توضیح ، کلید خارجی (\underline{کد طبابت}  ،\underline{ کد پرستاری})
		\item دکتر : تخصص ، 
		\underline{کد طبابت }
		، جنسیت ، نام ، نام خانوادگی 
		\item روزهای کاری  : نام ، 
		\underline{\lr{ID} }
		\item دارو  : 
		\lr{\underline{ID}}
		، نام ، شرکت تولید کننده ، تاریخ انقضاء ، قیمت 

	
		\end{enumerate}
	\end{center}
جداول مربوط به روابط به صورت زیر می باشد :
\begin{center}
	\begin{enumerate}
	
	\item نظارت 
	(بین پرستار و اتاق بستری ) :
	کلید خارجی ( 
	\underline{کد پرستاری}  ، \underline{شماره اتاق})
	

\item کار می کند  
(بین دکتر و روزهای کاری) : 
کلید خارجی (
\underline{کد طبابت} ،
\underline{\lr{ID}
روزهای کاری })
\item می خواهد 
(بین دارو و ویزیت ) : 
کلید خارجی (
\underline{\lr{ID}
دارو  }،
\underline{\lr{ID}
 ویزیت })



	\end{enumerate}
\end{center}
\subsection*{سیستم رزرو اتاق خانه معلم گیلان : }
جداول به شرح زیر می باشند  : 
\begin{center}
	\begin{enumerate}
		\item اتاق  : 
		\underline{شماره اتاق }
		، ظرفیت اتاق ، مبلغ 
		\item درخواست رزرو  : 
		\underline{شماره درخواست }
		، وضعیت ، تاریخ ثبت ، کلید خارجی (\underline{کد کارمندی} ، \underline{شماره اتاق} ، \underline{شماره پرسشنامه})
		\item پرسشنامه : 
		\underline{شماره پرسشنامه }
		، امتیاز 
		\item معلم : 
		‌\underline{کد کارمندی }
		، نوع استخدام ، نام ، نام خانوادگی ، سن ، سال تولد ، میزان حقوق ، سابقه تدریس ، سال استخدام 
	\end{enumerate}
\end{center}
\end{document}